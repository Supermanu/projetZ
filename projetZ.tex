\documentclass[11pt]{article}
\usepackage[utf8]{inputenc} 
%\usepackage[latin1]{inputenc}
\usepackage[T1]{fontenc}
\usepackage[french]{babel}
\usepackage{lmodern}

%\usepackage{setspace}
\usepackage[bottom=3cm, top=3cm,left=4cm, right=4cm]{geometry}

\usepackage{float}
\usepackage{graphicx}
%\usepackage{wrapfig}

\usepackage{amsthm}
\usepackage{amsmath}
\usepackage{amssymb}
\usepackage{mathrsfs}

%\usepackage[overload]{empheq}


\newtheorem{champ_parallele}{Définition}
\newtheorem{geodesique}{Définition}
\newtheorem{unicite}{Lemme}
\newtheorem{isometrie}{Définition}

\pagestyle{headings}

\makeatletter
\def\clap#1{\hbox to 0pt{\hss #1\hss}}%
\def\ligne#1{%
\hbox to \hsize{%
\vbox{\centering #1}}}%
\def\haut#1#2#3{%
\hbox to \hsize{%
\rlap{\vtop{\raggedright #1}}%
\hss
\clap{\vtop{\centering #2}}%
\hss
\llap{\vtop{\raggedleft #3}}}}%
\def\bas#1#2#3{%
\hbox to \hsize{%
\rlap{\vbox{\raggedright #1}}%
\hss
\clap{\vbox{\centering #2}}%
\hss
\llap{\vbox{\raggedleft #3}}}}%
\def\maketitle{%
\thispagestyle{empty}\vbox to \vsize{%
\haut{}{\@blurb}{}
\vfill
\vspace{1cm}
\begin{flushleft}
\usefont{OT1}{ptm}{m}{n}
\huge \@title
\end{flushleft}
\par
\hrule height 4pt
\par
\begin{flushright}
\usefont{OT1}{phv}{m}{n}
\Large \@author
\par
\end{flushright}
\vspace{1cm}
\vfill
\vfill
\bas{}{\@location, \@date}{}
}%
\cleardoublepage
}
\def\date#1{\def\@date{#1}}
\def\author#1{\def\@author{#1}}
\def\title#1{\def\@title{#1}}
\def\location#1{\def\@location{#1}}
\def\blurb#1{\def\@blurb{#1}}

\blurb{}
\makeatother

\title{Calorimétrie et boson de Higgs}
\author{ Maëlle Joveneau \\ Manuel Tondeur \\ Brieuc François }
\date{Année académique 2011-2012}
\location{Louvain - La - Neuve}

\blurb{%
Université Catholique de Louvain\\
Faculté des Sciences\\
\textbf{LPHY2135-Computing et Méthodes numériques en physique des particules}\\
Bruno Giacomo Luca}% 

\setlength{\parskip}{1ex plus .6ex minus .6ex}
\renewcommand\arraystretch{1.5}
%\onehalfspacing

\begin{document}

	\maketitle
		
	\tableofcontents

	\newpage
	
			\section{Introduction}
	Afin de d\'ecouvrir les principes de bases de Geant4, un framework conçu
pour la physique des particules, nous avons cr\'eé un programme permettant de
d\'eterminer la masse du Higgs en suivant la d\'esint\'egration de celui-ci en
deux $Z^0$ qui eux-m\^emes se d\'esint\`egrent en $e^-$ $e^+$.

Pour ce faire nous avons d'abord dû construire un d\'etecteur qui soit efficace
pour notre d\'esint\'egration. Nous avons d\'etermin\'e que les \'el\'ements
n\'ecessaires pour ce d\'etecteur \'etaient : des calorim\`etres
\'electromagn\'etiques et des trackers. Nous avons donc dû utiliser les classes
de Geant4 pertinentes pour nos besoins.
 
Nous avons alors g\'en\'er\'e nous-m\^eme les diff\'erentes particules qui
apparaissent dans l'\'ev\`enement qui nous int\'eresse. Pour donner aux
diff\'erentes particules leur masse, leur \'energie, et/ou leur impulsion, nous
avons dû faire appel \`a des fonctions de distribution vari\'ees qui nous ont
permises de mettre en pratique la th\'eorie des variables aléatoires dont la
méthode "Monte Carlo".
 
A partir des classes abstraites de Geant4, nous avons défini les interactions
possibles entre les particules et les \'el\'ements pr\'esents au sein du
d\'etecteur. Cela fait il nous fallait reconstruire la trajectoire des
\'electrons \`a partir des informations fournies par les trackers et les
calorim\`etres. Avec celles-ci, nous avons pu d\'eterminer l'\'energie et
l'impulsion des quatre \'electrons qui nous donnent la masse du Higgs par la
relation de la masse invariante suivante : 
 
\begin{equation}
M_H^2=E_H^2-p_H^2=(E_1+E_2+E_3+E_4)^2-(\vec{p}_1+\vec{p}_2+\vec{p}_3+\vec{p}_4)^2.
\end{equation} 
    
Afin d'inclure les erreurs dues aux bruits et celles dues \`a ?????(erreur
stochastique)??????, nous avons du...


		    \section{Descriptifs et méthodes}

\subsection{Description du détecteur}

Pour la construction du détecteur nous avons opté pour une géométrie cylindrique
de type CMS que nous avons bien entendu simplifiées pour ne garder que les
éléments utiles à notre simulation. Le détecteur est donc constitué de plusieurs
trackers en forme de tubes imbriqués et d'un calorimètre
électromagnétique granulaire qui englobe le tout. Ces éléments sont placés dans
le "world volume" que nous avons choisi de remplir d'air pour se rapprocher de
la réalité.

Comme dans le CMS, les trackers sont fait de silicium et mesurent 13.6 m de
long.Ils sont au nombre de treize, nous en avons mis trois très proches du
centre, le premier étant à 8 cm du vertex et les deux autres espacé de 1 cm du
premier. Les dix autres trackers se trouvent espacés chacun de 10 cm, le premier
ayant un rayon 20 cm. Bien qu'il nous fallait plusieurs trackers pour obtenir
une bonne résolution sur la direction des particules, ce choix particulier est
fait pour se rapprocher du CMS. Étant donné que notre projet ne nécessite
aucunement un champ magnétique, nous avons simplement choisi de ne pas en
mettre pour ne pas nous compliquer la tâche inutilement. 

Le bloc de calorimètres est un gros tube de 14 m de long dont
l'épaisseur s'étend de 1.5 m du centre à 3 m du centre. Il est uniquement
composé de calorimètres électromagnétiques car l'événement d'intérêt n'implique
ni muon ni hadron. L'épaisseur fût choisie pour éviter toute perte d'énergie.

Je ferai la segmentation à la fin pcq ca peut encore changer
Je dois encore dire pk on a mit de si grosse cellules si on laisse commme ça et
l'iodure de Cesium c est comme dans CMS j'imagine?

La suite au prochain épisode 





		
\end{document}










			